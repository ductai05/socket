\newpage
\section{Các thư viện và công nghệ}
% Đồ án đã sử dụng các thư viện, công nghệ, ngôn ngữ lập trình như sau:

\subsection{C++} 
% \paragraph{Ngôn ngữ lập trình C++} 
\paragraph{}{\textbf{C++} \cite{c++} là một ngôn ngữ lập trình đa năng bậc cao được tạo ra bởi Bjarne Stroustrup như một phần mở rộng của ngôn ngữ lập trình C. Ngôn ngữ đã được mở rộng đáng kể theo thời gian và C++ hiện đại có các tính năng: lập trình tổng quát, lập trình hướng đối tượng, lập trình thủ tục, ngôn ngữ đa mẫu hình tự do có kiểu tĩnh, dữ liệu trừu tượng, và lập trình đa hình, ngoài ra còn có thêm các tính năng, công cụ để thao tác với bộ nhớ cấp thấp.}
\paragraph{Phiên bản:}{Đồ án sử dụng phiên bản C++17}

\subsection{OpenSSL} 
% \paragraph{Thư viện phần mềm OpenSSL} 
\paragraph{}{\textbf{OpenSSL} \cite{openssl} là một thư viện phần mềm cho các ứng dụng bảo mật truyền thông qua mạng máy tính, giúp chống nghe trộm và xác thực danh tính của máy chủ hoặc bên giao tiếp. OpenSSL sử dụng các thuật toán mã hóa để bảo vệ dữ liệu khỏi việc bị đánh cắp và các cơ chế xác thực để đảm bảo bạn đang kết nối với đúng máy chủ hoặc client. OpenSSL được sử dụng rộng rãi trong các máy chủ web internet, phục vụ phần lớn tất cả các trang web.}
\paragraph{Phiên bản}{Đồ án sử dụng phiên bản Win64 OpenSSL v3.4.0 Light \cite{openssl340}}

\subsection{cURL} 
% \paragraph{Thư viện cURL} 
\paragraph{}{\textbf{cURL} \cite{curl} là một dự án phần mềm máy tính cung cấp thư viện (libcurl) và công cụ dòng lệnh (curl) để truyền dữ liệu bằng nhiều giao thức khác nhau. cURL được phát hành lần đầu tiên vào năm 1997. cURL là chữ viết tắt của "Client URL". Nhà phát triển ban đầu và chính của cURL là Daniel Stenberg, nhà phát triển phần mềm người Thụy Điển.}

\paragraph{Phiên bản}{Đồ án sử dụng cURL có sẵn trên hệ điều hành Windows, phiên bản \texttt{>=} \textbf{ 8.9.1}.}

\subsection{windows.h, winsock2.h} 
\paragraph{}{\textbf{windows.h} \cite{windows.h} là một file header mã nguồn mà Microsoft cung cấp để phát triển các chương trình truy cập Windows API (WinAPI) thông qua cú pháp ngôn ngữ C. Nó khai báo các hàm WinAPI, các kiểu dữ liệu liên quan và các macro thông dụng.} 
\paragraph{}{\textbf{winsock2.h} \cite{windows.h} là một file header mã nguồn giúp ta sử dụng Windows Sockets API (Winsock). Winsock định nghĩa cách mà phần mềm truy cập các dịch vụ mạng. Winsock xác định giao diện chuẩn giữa một ứng dụng client TCP/IP và các tầng phía dưới trong chồng giao thức TCP/IP.}


\subsection{SDL} 
% \paragraph{Thư viện SDL} 
\paragraph{}{\textbf{SDL} \cite{SDL} (Simple DirectMedia Layer) là một thư viện đa nền tảng được thiết kế để giúp tạo ra các ứng dụng đa phương tiện như trò chơi và chương trình đồ họa. Đồ án sử dụng thư viện \texttt{SDL2}, \texttt{SDL2\_image}, \texttt{SDL2\_ttf}, \texttt{SDL2\_mixer} cùng với Visual Studio để biên dịch và xây dựng ứng dụng.


